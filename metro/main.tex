% !TeX program = xelatex
\documentclass[a4paper, 12pt]{article}
\usepackage{/home/lajope/.custom/tex/preamble}

\title{Метрология}
\begin{document}

\maketitle
\newpage

\section{Физические величины, измерительные шкалы и системы единиц.}
\subsection{Основные понятия и определения}
\begin{itemize}
	\item Метрология - наука об измерениях, методах и средствах обеспечения их
	      едининства и способах достижения требуемой точности.
	\item Физическая величина - это свойство, общее в качественном отношении ко
	      многим физическим объектам, но в количественном отношении
	      индивидуальное для каждого физического объекта.
	\item Единица величины - фиксированное значение величины, которое принято
	      за единицу данной величины и применяется для количественного выражения
	      однородных с ней величин.
	\item Измерение - совокупность операций, выполняемых для определения
	      количественного значения величины.
	\item Качественной характеристикой измеряемых величин является размерность
	      (обозначается DIM (заглавные буквы латинского алфавита)).
	      \begin{center}
		      \( DIM(l)= L \)

		      \( DIM(m)= M \)

		      \( DIM(t)= T \)

		      ...
	      \end{center}
	      Размерность поизводных физических величин выражается через размерности
	      основных физических величин с помощью степенного одночлена.
	      \[
		      DIM(Q) = L^{\alpha} \cdot M^{\beta} \cdot T^{\gamma} \cdot \ldots
	      \]
	      где \( L, M, T \) - размерности основных физических величин\\
	      а \( \alpha, \beta, \gamma \) - показатели размерности

	      Эти показатели могут мыть целыми или дробными, положительными или
	      отрицательными, включая 0. Если все показатели равны нулю, то такая
	      величина является безразмерной.

	      Безразменая величина может быть относительной (коэффициент полезного
	      действия)

	      Если взять логорифм, то получим логорифмическую величину.
	      Пример: \[ 1 \text{Б (Белл)} = \log_{10}{\frac{P_1}{P_0}}\]
	      \[ P_1 = 10 \cdot P_0 \]
	      \[ 1 \text{Б} = 10 \text{дБ} \]


	      \[ 1 \text{Б} = \log_{10}{\frac{\frac{U_1^2}{R}}{\frac{U_0^2}{R}}} = 20
		      \cdot \log_{10}{\frac{U_1}{U_0}} \text{(дБ)}\]
	      \[ P = \frac{U^2}{R} \]
	      \begin{itemize}
		      \item Декада:
		            \[ 1 \text{ декада } = \log_{10}{F_2 / F_1} \]
		            \[ F_2 = 10 \cdot F_1 \]
		      \item Октава:
		            \[ 1 \text{ октава } = \log_{2}{F_2 / F_1} \]
		            \[ F_2 = 2 \cdot F_1 \]

		            \[
			            A = 20 \cdot \log_{10}{\frac{U_1}{U_2}} \text{ (дБ)} = \sqrt{P
				            \cdot R }= \sqrt{0.6} = 0.755 \text{ (В)}
			            .\]
		            \[
			            P = \frac{U^2}{R}
			            .\]
		            \[
			            R = 600 \text{ (Ом)}
			            .\]
		            \[
			            P = 1 \text{ (мВт)}
			            .\]
	      \end{itemize}
	\item Количественной характеристикой физических величин служит размер. Размер - это
	      количественное содержание в данном объекте свойств соответствующих понятию
	      "физическая величина".

	      Уравнение измерения \( Q = q[Q] \)\\
	      где \( q[Q] \) - значение физической величины\\
	      \( [Q] \) - единица измерения
	\item	Однозначность измерения работает для всех, кроме некоторых (например,
	      твердость тела)
\end{itemize}

\subsection{Измерительные шкалы}
\textbf{Шкала измерений - это упорядоченная совокупность значений физической величины,
которая служит основой для её измерения.}

Виды шкал:
\begin{itemize}
	\item Шкала наименований - это качественная шкала. На ней нет ни размеров,
    ничего другого. (Атлас цветов, примеры обоев)
	\item Шкала порядка - Упорядоченная совокупность размеров, каждый из которых
    больше предыдущего. \( Q_1 < Q_2 < Q_3 < Q_4... \)

    На шкале порядка не определены никакие математические действия за
    исключением логических операций (больше, меньше, равно) (пример: шкала
    измерения скорости ветра по наблюдениям за паром в обычной жизни).

    Результатом измерения по шкале порядка является решение. Решение может
    быть неверным, если \( Q_i \approx Q_j \).\\
    \[ P_\text{ошиб} = P_I + P_{II} + \ldots \]
    \item Шкала интервалов.
      \[ \Delta Q = Q_i - Q_j \]\\
      где \( Q_j \) - опорное значение.

      На шкале интервалов определены такие математические действия, как сложение и
      вычитание. Шкала интервалов имеет условное нулевое значение.
    \item Шкала отношений
      \[ q = \frac{Q}{[Q]} \] 

      Самая информативная из всех шкал. На ней определены все математические
      действия.

\end{itemize}
\subsection{Международная система единиц физических величин}

11 международная конференция по мерам и весам в 1960-м году: была принята
Система СИ. Изначально включала 7 основных величин и 2 дополнительные:
радиан и стерадиан. Потом 2 дополнительные убрали.

3 основные единицы физики раньше:
\begin{itemize}
  \item сгс - сантиметр грамм секунда
  \item мкгса - метр килограмм градус сила ампер
  \item мтс - метр тонна секунда
\end{itemize}

Есть ещё и другие внесистемные величины, действующие наравне единицам СИ.
Пример: минута, час, день, градус, гектары, сотки,

% end section

\end{document}
