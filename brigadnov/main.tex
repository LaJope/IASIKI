% !TeX program = xelatex
\documentclass[12pt]{article}
\usepackage{/home/lajope/.custom/tex/preamble}
\graphicspath{{./pics/}}

\title{Информационные технологии}
\begin{document}

\maketitle
\newpage

\section{Лекция 1}

\subsection{Интерполяция}
\begin{center}
	\includegraphics[scale=0.2]{lec1/pic1.jpg}
\end{center}

На графике даны точки:
\begin{equation*}
	\begin{cases}
		M_k (x_k, y_k), \\
		k = \overline{1 \ldots m}
	\end{cases}
\end{equation*}

Задача: соединить узлы (точки), что решает интерполирование.
Необходимо подобрать простейшую функцию, которая соединяет все точки.

Полином степени \( n \):
\[  P_n = a_0 + a_1 \cdot x + a_2 \cdot x^2 + ... + a_n \cdot x^n \]

Количество коэфициентов:
\( n + 1 \hspace{10pt} (a_0, a_1, a_2, \ldots , a_n) \)

Нужно найти полином, который проходит через все узлы, с наименьшим количеством
коэфициентов.
\begin{equation*}
	\begin{cases}
		P_n(x_k) = y_k, \\
		k= \overline{1 \ldots m}
	\end{cases}
\end{equation*}

Для каждой точки на графике получится уравнение:
\begin{equation*}
	\begin{cases}
		a_0 + a_1 \cdot x_1 + a_2 \cdot x_1^2 + ... + a_n \cdot x_1^n = y_1 \\
		a_0 + a_1 \cdot x_2 + a_2 \cdot x_2^2 + ... + a_n \cdot x_2^n = y_2 \\
		\ldots \ldots \ldots                                                \\
		a_0 + a_1 \cdot x_m + a_2 \cdot x_m^2 + ... + a_n \cdot x_m^n = y_m
	\end{cases}
\end{equation*}

\( (a_0, a_1, a_2, \ldots , a_n) \) - неизвестные.

\( ((x_1, y_1), (x_2, y_2), (x_3, y_3), \ldots (x_m, y_m)) \) -
заданные точки.

\vspace{10pt}
Это \textbf{система линейных алгебраических уравнений}.
\vspace{10pt}

\begin{tabularx}{\textwidth}{XXX}
	{
		\centering
		Вектор неизвестных:
		\begin{equation*}
			\begin{cases}
				a_0    \\
				a_1    \\
				a_2    \\
				\ldots \\
				a_n
			\end{cases}
		\end{equation*}
	} &
	{
			\centering
			Вектор числовых значений:
			\begin{equation*}
				\begin{cases}
					y_0    \\
					y_1    \\
					y_2    \\
					\ldots \\
					y_n
				\end{cases}
			\end{equation*}
	} &
	{
			\centering
			Матрица коэфициентов:
			\begin{equation*}
				\begin{cases}
					1 \hspace{4pt} x_1 \hspace{4pt} x_1^2 \hspace{4pt} \ldots \hspace{4pt} x_1^n \\
					1 \hspace{4pt} x_2 \hspace{4pt} x_2^2 \hspace{4pt} \ldots \hspace{4pt} x_2^n \\
					\ldots                                                                       \\
					1 \hspace{4pt} x_m \hspace{4pt} x_m^2 \hspace{4pt} \ldots \hspace{4pt} x_m^n
				\end{cases}
			\end{equation*}
			Матрица должна быть квадратной (\( n+1 \) - столбцов, \( m \) - строк, \( n + 1 = m \))
		}
\end{tabularx}


Степень полинома должна быть равна \( n = m - 1 \)

\( x_i \neq x_j \) (все узлы различны) \( \implies det(A) \neq 0 \)
(матрица невырождена) \( \implies \) имеется единственное решение при
\( n = m - 1 \implies \) полином, проходящий через эти точки, лишь один.

\vspace{10pt}

В итоге, мы получили:

\vspace{10pt}

\begin{tabularx}{\textwidth}{X|X|X}
	{
		\centering
		\( m = 1 \) (Если 1 эксперементальная точка то это полином 0 степени):
		\[ P_0(x) = a_0 \]
		\includegraphics[scale=0.1]{lec1/pic2.jpg}
	}
	 &
	{
			\centering
			\( m = 2 \):
			\[ P_1(x) = a_0 + a_1 \cdot x \]
			\includegraphics[scale=0.1]{lec1/pic3.jpg}
		}
	 &
	{
			\centering
			\( m = 3 \):
			\[ P_2(x) = a_0 + a_1 \cdot x + a_2 \cdot x^2 \]
			\includegraphics[scale=0.1]{lec1/pic4.jpg}
		}
\end{tabularx}

\vspace{10pt}

(Решение этой задачи в матлабе: \( a = A \backslash B \))
% subsection Интерполяция

\subsection{Интерполиционный полином в форме Лагранжа}

\textbf{Задача та же}

Обозначение формулы Лагранжа:
\[
	L_n(x) = \sum_{k = 1}^{m} y_k \cdot \omega_k(x)
\]
\[
	L_{m-1}(x) = \sum_{k = 1}^{m} y_k \cdot \omega_k(x) \hspace{10pt} (n = m - 1)
\]

Условия, которым должна соответствовать функция \( \omega_k(x) \):
\begin{itemize}
	\centering
	\item
	      \(
	      \begin{cases}
		      \omega_k(x_i) = 0 \text{ где } i \neq k \\
		      \omega_k(x_k) = 1
	      \end{cases}
	      \)
	\item
	      \( \omega_k(x) = P_{m-1}(x) \)
\end{itemize}

Функция, удовлетворяющяя этим условиям:
\[
	\omega_k(x) = \frac
	{(x-x_1)(x-x_2)\ldots(x-x_{k-1})(x-x_{k+1})\ldots(x-x_m)}
	{(x_k-x_1)(x_k-x_2)\ldots(x_k-x_{k-1})(x_k-x_{k+1})\ldots(x_k-x_m)}
\]

Получаем:
\[
	L_{m-1}(x) =
	\sum_{k = 1}^{m} \left(y_k \cdot
	\prod_{\substack{i=1 \\ i \neq k}}^{m}
	\frac{(x-x_i)}{(x_k-x_i)}\right)
\]

Найдём вид формулы Лагранжа для определённых \( m \):
\begin{itemize}
	\item \( m = 2: \hspace{32pt} P_1(x) = a_0 + a_1 \cdot x \)
	      \[
		      L_1(x) = y_1 \cdot \left(\frac{x-x_2}{x_1-x_2}\right) +
		      y_2 \cdot \left(\frac{x-x_1}{x_2-x_1}\right) =
		      \left(\frac{y_2-y_1}{x_2 - x_1}\right) x +
		      \left(\frac{y_1 \cdot x_2 - y_2 \cdot x_1}{x_2 - x_1}\right)
	      \]
	      \[ a_1 = \frac{y_2-y_1}{x_2 - x_1} \]
	      \[ a_0 = \frac{y_1 \cdot x_2 - y_2 \cdot x_1}{x_2 - x_1} \]
	\item \( m = 3: \hspace{32pt} P_2(x) = a_0 + a_1 \cdot x + a_2 \cdot x^2 \)
	      \[
		      L_2(x) = y_1 \cdot \frac{(x-x_2)(x-x_3)}{(x_1-x_2)(x_1-x_3)} +
		      y_2 \cdot \frac{(x-x_1)(x-x_3)}{(x_2-x_1)(x_2-x_3)} +
		      y_3 \cdot \frac{(x-x_1)(x-x_2)}{(x_3-x_1)(x_3-x_2)}
	      \]
\end{itemize}

Пример (m = 3):

\vspace{4pt}

\includegraphics[scale=0.2]{lec1/pic5.jpg}

Даны 3 точки. (0, 0), (h, 1), (1, 0).
Значение функции в точке h было измерено с прогрешностью \( \delta \)
(\( |\delta| \ll 1 \)).

Нужно найти, при каком значении h, данная погрешность имеет минимальное
влияние на конечный полином.

\begin{itemize}
	\item Сначала не берём во внимание погрешность.
	      \[
		      L_2(x) = 1 \cdot \frac{(x-0)(x-1)}{(h-0)(h-1)} = \frac{x(1-x)}{h(1-h)}
	      \]
	      \[
		      L_2\left(\frac{1}{2}\right) = \frac{1}{4h(1-h)}
	      \]
	\item Тепер, учитывая погрешность.
	      \[
		      \overline{L_2}(x) = (1 + \delta) \cdot
		      \frac{x(1-x)}{h(1-h)}
	      \]
	      \[
		      \overline{L_2} - L_2 = \delta \frac{x(1-x)}{h(1-h)}
	      \]
	      При \( x = \frac{1}{2} \), если \( h \sim 0 \) или \( h \sim 1 \implies
	      |\overline{L_2} - L_2| \gg |\delta| \implies \) погрешность сильно
	      влияет на конечный полином.

	      При \( h = \frac{1}{2} \implies |\overline{L_2} - L_2| = |\delta|
	      \implies \) погрешность мало влияет на полином (система устойчива).
\end{itemize}
% Subsection Интерполиционный полином в форме Лагранжа

\subsection{Метод аппроксимации.}
\subsubsection{Облако экспериментальных данных.}
\begin{wrapfigure}{l}{0.4\textwidth}
	\includegraphics[scale=0.15]{lec1/pic6.jpg}
\end{wrapfigure}

Дано множество (облако) точек:
\[ M_k(x_k;y_k), \hspace{4pt} k=\overline{1 \ldots m} \]
\[ m \gg 1 \]

Вопрос: какая представлена зависимость на графике?

Уравнение аппроксимации, завиящей от параметров \( A, B, C, \ldots \):
\[ y = f(x, A, B, C, \ldots) \]

\vspace{10pt}
Для нахождения хорошей функции аппроксимации \( y = c_0 + c_1 \cdot x + c_2
\cdot x^2 + \ldots  \)
необходимо, чтобы разница, между данной функцией аппроксимации и
эксперементальным значением функции в точке \( x_k \) была наименьшей.

\textbf{Метод наименьших квадратов:}
\[
	S(c_1; c_2; \ldots; c_m) =
	\sum_{k=1}^{m}(f(x_k; c_1; c_2; \ldots ; c_m) - y_k)^2 \to
	\substack{min\\(c_1; c_2; \ldots ; c_m)}
\]

% Subsubsection Облако экспериментных данных.

\subsubsection{Линейная регрессия:}
\hspace{40pt} \( y = c_0 + c_1 \cdot x \)

\[
	S(C_1, C_2) = \sum_{k=1}^m \left( C_1 + C_2 \cdot x_k - y_k \right)^2
	\to \substack{min\\(C_1; C_2)}
\]

Необходимое и достаточное условие экстремума:
\begin{equation*}
	\begin{cases}
		\frac{dS}{dC_1} = 0 \\
		\frac{dS}{dC_2} = 0
	\end{cases}
	\implies
	\begin{cases}
		\frac{dS}{dC_1} =
		\sum_{k=1}^m 2 \cdot \left( C_1 + C_2 \cdot x_k - y_k \right) = 0 \\
		\frac{dS}{dC_2} =
		\sum_{k=1}^m 2 \cdot \left( C_1 + C_2 \cdot x_k - y_k \right) \cdot x_k = 0
	\end{cases}
\end{equation*}

\begin{equation*}
	\begin{cases}
		C_1 \sum_{k=1}^{m}1 + C_2 \sum_{k=1}^{m}x_k = \sum_{k=1}^{m}y_k \\
		C_1 \sum_{k=1}^{m}x_k + C_2 \sum_{k=1}^{m}x_k^2 = \sum_{k=1}^{m}x_k \cdot y_k
	\end{cases}
\end{equation*}

\[
  a_{11} = m; \hspace{6pt} a_{12} = a_{21} = \sum_{k=1}^{m}x_k;
	\hspace{6pt} a_{22} = \sum_{k=1}^{m}x_k^2; \hspace{6pt}
	b_1 = \sum_{k=1}^{m}y_k; \hspace{6pt} b_2 = \sum_{k=1}^{m} x_k \cdot y_k
\]

\begin{equation*}
	\begin{cases}
		a_{11} C_1 + a_{12} C_2 = b_1 \\
		a_{21} C_1 + a_{22} C_2 = b_2
	\end{cases}
\end{equation*}

Пример \( m = 3 \):

\begin{wrapfigure}{l}{0.55\textwidth}
  \includegraphics[scale=0.2]{lec1/pic7.jpg} 
\end{wrapfigure}

Даны точки: \( (0,0); (h, 1); (1, 0) \)\\

По формулам:\\
\( a_{11} = 3 \)\\
\( a_{12} = a_{21} = 1 + h \)\\
\( a_{22} = 1 + h^2 \)\\
\( b_1 = 1 \)\\
\( b_2 = h \)
\begin{equation*}
  \begin{cases}
    3 C_1 + (1 + h) C_2 = 1\\
    (1 + h) C_1 + (1 + h^2) C_2 = h
  \end{cases}
\end{equation*}\\

Найдём решение данной системы матричным методом:
\[
  det(A) = \left|
  \begin{matrix}
    3 & 1+h\\
    1+h & 1+h^2
  \end{matrix}
  \right| = 3 + 3 h^2 - (1 + h^2) = 2 (1 - h + h^2) \neq 0
\] 
\[
  \Delta_1 = \left|
  \begin{matrix}
    1 & 1 + h\\
    h & 1 + h^2
  \end{matrix}
  \right| = 1 + h^2 - h - h^2 = 1 - h
\] 
\[
  \Delta_2 = \left|
  \begin{matrix}
    3 & 1\\
    1+h & h
  \end{matrix}
  \right| = 3 h - 1 - h = 2 h - 1
\] 

\[ C_1 = \frac{\Delta_1}{det(A)} = \frac{1 - h}{2 (1 - h + h^2)} \] 
\[ C_2 = \frac{\Delta_2}{det(A)} = \frac{2 h - 1}{2 (1 - h + h^2)} \] 

При погрешности измерени функции в точке h, метод наименьших квадратов
обеспечивает устойчивость системы.
% Subsubsection Линейная регрессия
% Subsection Метод аппроксимации. Облако экспериментных данных.
% Section Лекция 1

\newpage

\section{Лекция 2}
\subsection{Численное интегрирование}
\subsubsection{Постановка задачи}
\textbf{Задача: найти значение определённого интеграла с любой определённой точностью}

\[
  S = \int_a^b f(x)dx
\] 

Первое: дискретизировать задачу

РИС 1

\[
  x_0 = a, x_n = b
\] 

\[
  h = \frac{b - a}{n}; n \gg 1
\]

РИС 2

Интеграл - площадь криволинейной трапеции. Нужно найти площадь.

Разбили, разрезали на полоски через \( x_i \)


Как решать: метод аппроксимации.

Рассмотрим элементарный отрезочек

РИС 3

\( k = \overline{1,\dots,n} \)
\[
  S = \int_{a}^{b} f(x)dx = \sum_{k=1}^{n} \int_{x_{k-1}}^{x_k} f(x)dx
\] 

Теперь нужно научиться приближённо считать элементарные интегралы

Формулы механических квадратур:

\[
  \int_{x_{k-1}}^{x_k} f(x)dx \approx \sum_{i=1}^m A_i f(x_i)
\] 

где m - определённый параметр,
A - коэффициент

% Subsubsection Постановка задачи

\subsubsection{Методы решения}
\begin{itemize}
  \item Методы Ньютона-Котеса:
        \begin{itemize}
          \item Берём \( m=1 \):

                Заменяем нашу функцию интерполяционным полиномом 0 степени
                \[
                    f(x) \approx L_0(x)         
                \] 
                Берём середину отрезка \( \overline{x_k} \)

                РИС 4

                Т.е. мы заменяем функцию кусочно-постоянной функцией
                (прямоугольники).

                В этом случае элементарная площадь равняется:

                \[
                  \int_{x_{k-1}}^{x_k} f(x)dx \approx
                  f(\overline{x}_k) \cdot h
                \] 
                \[
                  S = \int_{a}^{b} f(x)dx \approx
                  h \cdot \sum_{k=1}^{n}f(\overline{x}_k)
                \] 

                Этот метод называется методом прямоугольников
                с выбором средней точки.  

                Если выбрать левый \( x_{k-1} \) за точку, то это метод
                левых прямоугольников и т.д.

                \textbf{Кусочно-линейная постоянная апроксимация}
          \item Берём \( m=2 \):
            \[
              f(x) = L_1(x)
            \] 

            РИС 5

            Заменяем полиномом 1 степени (прямой)

            \[
              \int_{x_{k-1}}^{x_k} f(x)dx \approx
              \int_{x_{k-1}}^{x_k} L_1(x)dx =
              h \cdot \left( \frac{f(x_{k-1}) + f(x_k)}{2} \right)
            \] 

            Это метод трапеции

            Кусочно-линейная непрерывная апроксимация

          \item Берём \( m=3 \) (Первый сделал Симпсон):

            РИС 6

            \[
              f(x) \approx L_2(x)
            \] 
            \[
              \int_{x_{k-1}}^{x_k} f(x)dx \approx
              \int_{x_{k-1}}^{x_k} L_2(x)dx =
              \frac{h}{6} \cdot (f(x-1) + 4 \cdot f(\overline{x_k}) + f(x_k))
            \] 
            Формула Симпсона (среднее взвешенное значение)

        \end{itemize}
        Как добиться, используя любую из этих формул, любой точности?  
        \begin{quotation}
          >Чем меньше интервальчик, тем лучше апроксимация.<
        \end{quotation}
        \[
          S_n \hspace{2pt} \substack{\longrightarrow \\ n \to \infty}
          \hspace{2pt} S
        \] 
        \[
          \abs{S_{2n} - S_n} \hspace{2pt}
          \substack{\longrightarrow \\ n \to \infty}
          \hspace{2pt} 0
        \] 

        Этот принцип положен в основу оценки точности приближения
        
        Берём \( n=2 \); \( S_2 \)

        Потом \( n=2 \cdot n \); \( S_4 \)

        Если разность меньше заданной точности, т.е.
        \[
          \abs{S_{2n} - S_n} < \epsilon
        \] 
        то ещё удваиваем n.

        \( \epsilon \) - заданная точность

        \begin{quotation}
          >Карл Фридрих Гаусс хорошенько поёрзал<
        \end{quotation}
  \item Методы Гаусса
      
    Идея: так же разбили на элементарные промежутки
    \[
      \int_{a}^{b} f(x) dx = \sum_{k=1}^{n} \int_{x_{k-1}}^{x_k} f(x)dx
    \] 
    где \( \displaystyle\int_{x_{k-1}}^{x_k} f(x)dx = S_k \) - элементарная площадь.
    \[
      S_k \approx \sum_{i=1}^{m} A_i \cdot f(x_i)
    \] 

    В формулах Ньютона-Котеса узлы фиксированны. 
    Гаусс задумался, можно ли пошевелить узлами, чтобы добиться наибольшей
    точности?

    Для этого он перевёл интегрирование в другую форму:
    \[
      \int_{x_{k-1}}^{x_k} f(x)dx = C \cdot \int_{-1}^{1} \phi(t)dt
    \] 
    Чтобы привести интеграл в данную форму, нужно преобразовать x в t линейной
    заменой, т.е.:
    \[
      [x_{k-1}; x_k] \longrightarrow [-1; 1]
    \] 
    *13
    
    Нужно подобрать m так, чтобы сумма в точности равнялась полиному степени 2m
    -1
    *14

    1) m = 1: *15

    Это доказывает, что метод с выбором средней точки абсолютно точен для
    любой линейной функции.

    *16

    2) m = 2: *17

\end{itemize}

% Subsubsection Методы решения
% Subsection Численное интегрирование
% Section Лекция 2

\end{document}
